%! Author = Kevin Lin
%! Date = 1/16/2026

% Preamble
\documentclass[11pt,a4paper,margin=1in]{article}

% Packages
\usepackage{graphicx}
\usepackage{float}

\title{HW 1: Project}
\author{Kevin Lin}
\date{1/16/2026}

% Document
\begin{document}
\maketitle

\section{}
I chose the medical insurance dataset, as healthcare costs and insurance coverage
in general is an important issue unique to the United States. While this 
rudimentary dataset only contains a few features and definitely won't make
any groundbreaking predictions, it is a good starting point to explore insurance
costs and how they vary across different demographics. In my undergraduate senior
year, I worked with a dataset from MIMIC-III, which contained de-identified health data
from critical care patients. I used that dataset to predict hospital readmission
rates with an AUC of ~0.8 matching a study done by Google. Hopefully I can find
some interesting trends in this dataset as well, or uncover some biases in 
insurance coverage and costs.

\section{}
There is no direct research article associated with this dataset. The dataset 
was curated specifically for educational purposes for Brett Lantz's book 
\textit{Machine Learning with R} from PACKT Publishing. However, Lantz 
acknowledges that the dataset was originally sourced ``using demographic 
statistics from the U.S. Census Bureau, and thus approximately reflect[s] 
real-world conditions.'' The variables in the dataset are as follows:

\begin{table}[H]
\centering
\resizebox{\textwidth}{!}{%
\begin{tabular}{|l|l|l|l|}
\hline
Data     & Description                                         & Variable Type & Units          \\ \hline
age      & age of primary benificiary, excluding anyone 64+    & integer       & years          \\ \hline
sex      & policy holder's gender (male or female)             & binary string & N/A            \\ \hline
bmi      & policy holder's body mass index                     & float         & kg/m           \\ \hline
children & number of children / dependents covered by the plan & integer       & \# of children \\ \hline
smoker   & whether the insured regularly smokes (yes or no)    & binary string & N/A            \\ \hline
region &
  \begin{tabular}[c]{@{}l@{}}beneficiary's place of residence in the US, divided into \\ 4 regions (northeast, northwest, southeast, southwest)\end{tabular} &
  categorical string &
  N/A \\ \hline
charges  & insurance charge amount in dollars                  & float         & \$USD          \\ \hline
\end{tabular}%
}
\end{table}

\end{document}